% Options for packages loaded elsewhere
\PassOptionsToPackage{unicode}{hyperref}
\PassOptionsToPackage{hyphens}{url}
%
\documentclass[
]{article}
\usepackage{amsmath,amssymb}
\usepackage{iftex}
\ifPDFTeX
  \usepackage[T1]{fontenc}
  \usepackage[utf8]{inputenc}
  \usepackage{textcomp} % provide euro and other symbols
\else % if luatex or xetex
  \usepackage{unicode-math} % this also loads fontspec
  \defaultfontfeatures{Scale=MatchLowercase}
  \defaultfontfeatures[\rmfamily]{Ligatures=TeX,Scale=1}
\fi
\usepackage{lmodern}
\ifPDFTeX\else
  % xetex/luatex font selection
\fi
% Use upquote if available, for straight quotes in verbatim environments
\IfFileExists{upquote.sty}{\usepackage{upquote}}{}
\IfFileExists{microtype.sty}{% use microtype if available
  \usepackage[]{microtype}
  \UseMicrotypeSet[protrusion]{basicmath} % disable protrusion for tt fonts
}{}
\makeatletter
\@ifundefined{KOMAClassName}{% if non-KOMA class
  \IfFileExists{parskip.sty}{%
    \usepackage{parskip}
  }{% else
    \setlength{\parindent}{0pt}
    \setlength{\parskip}{6pt plus 2pt minus 1pt}}
}{% if KOMA class
  \KOMAoptions{parskip=half}}
\makeatother
\usepackage{xcolor}
\usepackage[margin=1in]{geometry}
\usepackage{color}
\usepackage{fancyvrb}
\newcommand{\VerbBar}{|}
\newcommand{\VERB}{\Verb[commandchars=\\\{\}]}
\DefineVerbatimEnvironment{Highlighting}{Verbatim}{commandchars=\\\{\}}
% Add ',fontsize=\small' for more characters per line
\usepackage{framed}
\definecolor{shadecolor}{RGB}{248,248,248}
\newenvironment{Shaded}{\begin{snugshade}}{\end{snugshade}}
\newcommand{\AlertTok}[1]{\textcolor[rgb]{0.94,0.16,0.16}{#1}}
\newcommand{\AnnotationTok}[1]{\textcolor[rgb]{0.56,0.35,0.01}{\textbf{\textit{#1}}}}
\newcommand{\AttributeTok}[1]{\textcolor[rgb]{0.13,0.29,0.53}{#1}}
\newcommand{\BaseNTok}[1]{\textcolor[rgb]{0.00,0.00,0.81}{#1}}
\newcommand{\BuiltInTok}[1]{#1}
\newcommand{\CharTok}[1]{\textcolor[rgb]{0.31,0.60,0.02}{#1}}
\newcommand{\CommentTok}[1]{\textcolor[rgb]{0.56,0.35,0.01}{\textit{#1}}}
\newcommand{\CommentVarTok}[1]{\textcolor[rgb]{0.56,0.35,0.01}{\textbf{\textit{#1}}}}
\newcommand{\ConstantTok}[1]{\textcolor[rgb]{0.56,0.35,0.01}{#1}}
\newcommand{\ControlFlowTok}[1]{\textcolor[rgb]{0.13,0.29,0.53}{\textbf{#1}}}
\newcommand{\DataTypeTok}[1]{\textcolor[rgb]{0.13,0.29,0.53}{#1}}
\newcommand{\DecValTok}[1]{\textcolor[rgb]{0.00,0.00,0.81}{#1}}
\newcommand{\DocumentationTok}[1]{\textcolor[rgb]{0.56,0.35,0.01}{\textbf{\textit{#1}}}}
\newcommand{\ErrorTok}[1]{\textcolor[rgb]{0.64,0.00,0.00}{\textbf{#1}}}
\newcommand{\ExtensionTok}[1]{#1}
\newcommand{\FloatTok}[1]{\textcolor[rgb]{0.00,0.00,0.81}{#1}}
\newcommand{\FunctionTok}[1]{\textcolor[rgb]{0.13,0.29,0.53}{\textbf{#1}}}
\newcommand{\ImportTok}[1]{#1}
\newcommand{\InformationTok}[1]{\textcolor[rgb]{0.56,0.35,0.01}{\textbf{\textit{#1}}}}
\newcommand{\KeywordTok}[1]{\textcolor[rgb]{0.13,0.29,0.53}{\textbf{#1}}}
\newcommand{\NormalTok}[1]{#1}
\newcommand{\OperatorTok}[1]{\textcolor[rgb]{0.81,0.36,0.00}{\textbf{#1}}}
\newcommand{\OtherTok}[1]{\textcolor[rgb]{0.56,0.35,0.01}{#1}}
\newcommand{\PreprocessorTok}[1]{\textcolor[rgb]{0.56,0.35,0.01}{\textit{#1}}}
\newcommand{\RegionMarkerTok}[1]{#1}
\newcommand{\SpecialCharTok}[1]{\textcolor[rgb]{0.81,0.36,0.00}{\textbf{#1}}}
\newcommand{\SpecialStringTok}[1]{\textcolor[rgb]{0.31,0.60,0.02}{#1}}
\newcommand{\StringTok}[1]{\textcolor[rgb]{0.31,0.60,0.02}{#1}}
\newcommand{\VariableTok}[1]{\textcolor[rgb]{0.00,0.00,0.00}{#1}}
\newcommand{\VerbatimStringTok}[1]{\textcolor[rgb]{0.31,0.60,0.02}{#1}}
\newcommand{\WarningTok}[1]{\textcolor[rgb]{0.56,0.35,0.01}{\textbf{\textit{#1}}}}
\usepackage{longtable,booktabs,array}
\usepackage{calc} % for calculating minipage widths
% Correct order of tables after \paragraph or \subparagraph
\usepackage{etoolbox}
\makeatletter
\patchcmd\longtable{\par}{\if@noskipsec\mbox{}\fi\par}{}{}
\makeatother
% Allow footnotes in longtable head/foot
\IfFileExists{footnotehyper.sty}{\usepackage{footnotehyper}}{\usepackage{footnote}}
\makesavenoteenv{longtable}
\usepackage{graphicx}
\makeatletter
\def\maxwidth{\ifdim\Gin@nat@width>\linewidth\linewidth\else\Gin@nat@width\fi}
\def\maxheight{\ifdim\Gin@nat@height>\textheight\textheight\else\Gin@nat@height\fi}
\makeatother
% Scale images if necessary, so that they will not overflow the page
% margins by default, and it is still possible to overwrite the defaults
% using explicit options in \includegraphics[width, height, ...]{}
\setkeys{Gin}{width=\maxwidth,height=\maxheight,keepaspectratio}
% Set default figure placement to htbp
\makeatletter
\def\fps@figure{htbp}
\makeatother
\setlength{\emergencystretch}{3em} % prevent overfull lines
\providecommand{\tightlist}{%
  \setlength{\itemsep}{0pt}\setlength{\parskip}{0pt}}
\setcounter{secnumdepth}{-\maxdimen} % remove section numbering
\ifLuaTeX
  \usepackage{selnolig}  % disable illegal ligatures
\fi
\IfFileExists{bookmark.sty}{\usepackage{bookmark}}{\usepackage{hyperref}}
\IfFileExists{xurl.sty}{\usepackage{xurl}}{} % add URL line breaks if available
\urlstyle{same}
\hypersetup{
  pdftitle={Analisis poblaciones de trucha en León},
  pdfauthor={Santiago Fraga Sáenz},
  hidelinks,
  pdfcreator={LaTeX via pandoc}}

\title{Analisis poblaciones de trucha en León}
\author{Santiago Fraga Sáenz}
\date{13/12/2023}

\begin{document}
\maketitle

{
\setcounter{tocdepth}{5}
\tableofcontents
}
\hypertarget{anuxe1lisis-y-observaciuxf3n-previa-de-los-datos-publicados-por-la-junta-de-castilla-y-leuxf3n}{%
\section{1. ANÁLISIS Y OBSERVACIÓN PREVIA DE LOS DATOS PUBLICADOS POR LA
JUNTA DE CASTILLA Y
LEÓN}\label{anuxe1lisis-y-observaciuxf3n-previa-de-los-datos-publicados-por-la-junta-de-castilla-y-leuxf3n}}

Preparamos los datos. Antes de comenzar con el análisis mostramos la
estructura de los datos una vez tabulados. Estos datos se han extraído
de los informes publicados por la Junta de Castilla Y León
correspondientes con las redes de vigilancia y seguimineto
\href{https://medioambiente.jcyl.es/web/es/caza-pesca/informes-seguimiento-control-poblaciones.html}{link}.
Se ha creado unn data frame para la provincia de León que comprende las
siguientes variables:

\begin{verbatim}
##       Estacion    Provincia   Gestion    Nivel        año         Biomasa     
##  LE-AREC-3:   8   Leon:1064   AREC: 24   1:104   2014   :133   Min.   : 0.01  
##  LE-AREC-7:   8               CCM :120   2:960   2015   :133   1st Qu.: 2.90  
##  LE001    :   8               CM  : 96           2016   :133   Median : 5.59  
##  LE002    :   8               CSM : 24           2017   :133   Mean   : 7.33  
##  LE003    :   8               EDS :  8           2018   :133   3rd Qu.:10.58  
##  LE004    :   8               L   :720           2019   :133   Max.   :38.12  
##  (Other)  :1016               V   : 72           (Other):266   NA's   :671    
##     Densidad        Peso_medio   
##  Min.   :0.0000   Min.   : 4.50  
##  1st Qu.:0.0800   1st Qu.:22.51  
##  Median :0.1500   Median :36.75  
##  Mean   :0.2204   Mean   :  Inf  
##  3rd Qu.:0.3200   3rd Qu.:54.66  
##  Max.   :2.0400   Max.   :  Inf  
##  NA's   :673      NA's   :673
\end{verbatim}

Las variables Estación, Provincia, Gestión, Nivel y año son de tipo
factor y el resto numéricas. Dentro de la gestión diferenciamos los
siguientes niveles en función del tramo en el que se situó la estación
de muestreo:

L= Tramo libre

CCM= Coto con muerte mixto

V= Vedado

CM= coto con muerte

AREC= Aguas en régimen especial controlado

ARE= Aguas en regimen especial

CSM= Coto sin muerte

Los datos de las variables métricas de las poblaciones de trucha son:

Biomasa; medida en gr/m2

Densidad; medido en indv/m2

Peso medio: Resultado de dividir la biomasa entre la densidad, medida en
gr/indv

\hypertarget{anuxe1lisis-de-la-variable-biomasa}{%
\section{2. ANÁLISIS DE LA VARIABLE
BIOMASA}\label{anuxe1lisis-de-la-variable-biomasa}}

Para dar respuesta a la pregunta que se plantea en esta investigación
que no es otra que ver si el cambio normativo introducido por la ley de
pesca de 2013 ha supuesto cambios detectables en las variables
poblacionales de trucha, tenemos que definir como, en función de los
datos existentes, vamos a comprobarlo. Debido a que los datos publicados
solo exponen la biomasa y la densidad como únicas variables obtenidas
para cada estación, se propone definir un análisis de comparación entre
grupos considerando el año 2014 como año de comparación ya que los
efectos del cambio de gestión derivados de la nueva ley aún serían
incipientes. De los modelos de gestión existentes el posible gran cambio
se produciría en los tramos libres ya que pasarían a ser gestionados en
el totalidad de las aguas trucheras como libres sin muerte. Observamos
en primer lugar la distribución de las variables poblacionales, como
variables respuesta, respecto a los grupos anuales y a los sistemas de
gestión.

\hypertarget{visualizaciuxf3n-de-los-datos}{%
\subsection{2.1 VISUALIZACIÓN DE LOS
DATOS}\label{visualizaciuxf3n-de-los-datos}}

\includegraphics{Informe_files/figure-latex/unnamed-chunk-4-1.pdf}

Representamos las variables para los tramos libres, donde es previsible
que se puedan observar con los efectos que se quieren comprobar, ya que
en ellos la gestión ha pasado de poder sacrificarse las capturas que
superasen la talla mínima sin un control de acceso del número de
pescadores a ser tramos donde el sacrificio está prohibido, y todo este
se produce por primera vez en el año 2014. Por ello este año es el año
de referencia con el que se compararán el resto de años para observar si
hay cambios significativos en las variables medidas.

\includegraphics{Informe_files/figure-latex/unnamed-chunk-5-1.pdf}

Se observa la existencia de posibles outliers que para la comprobación
de las comparaciones y realizar las comprobaciones de hipótesis deberá
ser tenida en cuenta para elegir la tecnica estadística. Determinamos a
continuación los valores medios para cada año para los tramos de pesca
libre, recogiéndose igualmente las desviaciones estandar de cada grupo.

\hypertarget{medias-y-desviaciones-de-la-variable-biomasa}{%
\subsection{2.2 MEDIAS Y DESVIACIONES DE LA VARIABLE
BIOMASA}\label{medias-y-desviaciones-de-la-variable-biomasa}}

Para los tramos libres de León, comprobamos los valores medias de
biomasa de trucha (en gr/m2) en los diferentes años, así como los
valores de la desviación estándar.

\begin{longtable}[]{@{}
  >{\raggedright\arraybackslash}p{(\columnwidth - 16\tabcolsep) * \real{0.0759}}
  >{\raggedleft\arraybackslash}p{(\columnwidth - 16\tabcolsep) * \real{0.1139}}
  >{\raggedleft\arraybackslash}p{(\columnwidth - 16\tabcolsep) * \real{0.1139}}
  >{\raggedleft\arraybackslash}p{(\columnwidth - 16\tabcolsep) * \real{0.1139}}
  >{\raggedleft\arraybackslash}p{(\columnwidth - 16\tabcolsep) * \real{0.1139}}
  >{\raggedleft\arraybackslash}p{(\columnwidth - 16\tabcolsep) * \real{0.1139}}
  >{\raggedleft\arraybackslash}p{(\columnwidth - 16\tabcolsep) * \real{0.1266}}
  >{\raggedleft\arraybackslash}p{(\columnwidth - 16\tabcolsep) * \real{0.1139}}
  >{\raggedleft\arraybackslash}p{(\columnwidth - 16\tabcolsep) * \real{0.1139}}@{}}
\caption{Valores medios y desviaciones estandar}\tabularnewline
\toprule\noalign{}
\begin{minipage}[b]{\linewidth}\raggedright
\end{minipage} & \begin{minipage}[b]{\linewidth}\raggedleft
2014
\end{minipage} & \begin{minipage}[b]{\linewidth}\raggedleft
2015
\end{minipage} & \begin{minipage}[b]{\linewidth}\raggedleft
2016
\end{minipage} & \begin{minipage}[b]{\linewidth}\raggedleft
2017
\end{minipage} & \begin{minipage}[b]{\linewidth}\raggedleft
2018
\end{minipage} & \begin{minipage}[b]{\linewidth}\raggedleft
2019
\end{minipage} & \begin{minipage}[b]{\linewidth}\raggedleft
2020
\end{minipage} & \begin{minipage}[b]{\linewidth}\raggedleft
2021
\end{minipage} \\
\midrule\noalign{}
\endfirsthead
\toprule\noalign{}
\begin{minipage}[b]{\linewidth}\raggedright
\end{minipage} & \begin{minipage}[b]{\linewidth}\raggedleft
2014
\end{minipage} & \begin{minipage}[b]{\linewidth}\raggedleft
2015
\end{minipage} & \begin{minipage}[b]{\linewidth}\raggedleft
2016
\end{minipage} & \begin{minipage}[b]{\linewidth}\raggedleft
2017
\end{minipage} & \begin{minipage}[b]{\linewidth}\raggedleft
2018
\end{minipage} & \begin{minipage}[b]{\linewidth}\raggedleft
2019
\end{minipage} & \begin{minipage}[b]{\linewidth}\raggedleft
2020
\end{minipage} & \begin{minipage}[b]{\linewidth}\raggedleft
2021
\end{minipage} \\
\midrule\noalign{}
\endhead
\bottomrule\noalign{}
\endlastfoot
Media & 3.670250 & 4.340357 & 7.080000 & 8.507143 & 4.878421 & 10.816000
& 8.015294 & 9.618250 \\
Desv & 3.675736 & 5.088801 & 3.464265 & 6.800465 & 3.858286 & 8.816594 &
5.917669 & 5.978341 \\
\end{longtable}

Se observa un aumento de los valores medios respecto a 2014.
Representamos la distribución mediante histograma de la variable biomasa

\includegraphics{Informe_files/figure-latex/unnamed-chunk-7-1.pdf}

\hypertarget{modelo-para-la-comparaciuxf3n-de-la-biomasa}{%
\subsection{2.3 MODELO PARA LA COMPARACIÓN DE LA
BIOMASA}\label{modelo-para-la-comparaciuxf3n-de-la-biomasa}}

Para la comparación de los grupos optamos en primer lugar por observar
los datos y ver los supuestos para adaptarlo a un modelo lineal.
Comprobamos la normalidad y la homocedasticidad. Primero los
visualizamos

\includegraphics{Informe_files/figure-latex/unnamed-chunk-8-1.pdf}
\includegraphics{Informe_files/figure-latex/unnamed-chunk-8-2.pdf}

Comprobamos la normalidad y homocedasticidad aunque es visible que no
parecen cumplirse ninguno de los 2 criterios

\begin{verbatim}
## 
##  Shapiro-Wilk normality test
## 
## data:  leon_tb_l$Biomasa
## W = 0.86414, p-value = 5.19e-14
\end{verbatim}

No se puede aceptar la la distribución normal de los datos

\begin{verbatim}
## Levene's Test for Homogeneity of Variance (center = median)
##        Df F value  Pr(>F)  
## group   7  2.4056 0.02137 *
##       240                  
## ---
## Signif. codes:  0 '***' 0.001 '**' 0.01 '*' 0.05 '.' 0.1 ' ' 1
\end{verbatim}

Tampoco se puede aceptar la homocedasticidad de los datos de biomasa
para los diferentes años. Por ello es preciso transformar los datos y
ver si con ello podemos asumir un modelo lineal. Para ello
determinaremos los valores de \(\lambda\) mediante la aplicación de una
transformación de Box Cox

\includegraphics{Informe_files/figure-latex/unnamed-chunk-11-1.pdf}

El valor de \(\lambda\) obtenido es 0.3434343 es el valor de la potencia
usada en la transformación de los datos.

Una vez transformado los datos visualizamos nuevamente
\includegraphics{Informe_files/figure-latex/unnamed-chunk-13-1.pdf}
\includegraphics{Informe_files/figure-latex/unnamed-chunk-13-2.pdf}

Visualmente ahora parece que los datos pueden cumplir alos criterios de
normalidad y homocedasticidad, pese a ello realizamos los test.

\begin{verbatim}
## 
##  Shapiro-Wilk normality test
## 
## data:  leon_tb_l$Biomasa
## W = 0.99583, p-value = 0.7467
\end{verbatim}

Se acepta la la distribución normal de los datos

\begin{verbatim}
## Levene's Test for Homogeneity of Variance (center = median)
##        Df F value Pr(>F)
## group   7  1.5997 0.1361
##       240
\end{verbatim}

Se acepta la homocedasticidad

Planteamos ahora un modelo lineal mixto. Esto es conveniente en nuestro
caso ya que podemos establecer una parte aleatoria del modelo con los
datos anidados por la variable Estación. Como término fijo utilizaríamos
la variable año, ya que se trata del factor principal que nos explica
precisamente si existen diferencias en los valores de la biomasa
considerando el año 2014 como el año que se produce el cambio de
política de gestión al establecer los libres como libres sin muerte.
Para la construcción del modelo seguimos el procedimiento que recomienda
Zuur et al.~(2009) y que se secuenciaría en fases:

1.Estructura aleatoria óptima. Usando un modelo saturado (beyond optimal
model), se determina la estructura óptima del componente aleatorio, la
cual no debe contener información que esté en la componente fija.
Debemos:

\begin{itemize}
\item
  construir un modelo saturado.
\item
  comparar modelos con distinta estructura aleatoria, mediante máxima
  verosimilitud restringida (REML).
\end{itemize}

2.Estructura fija óptima. Una vez encontramos la estructura aleatoria
óptima, podemos encontrar la estructura fija óptima. Comparamos los
modelos anidados mediante máxima verosimilitud (ML), manteniendo la
misma estructura aleatoria.

3.Ajuste del modelo final con REML.

\begin{Shaded}
\begin{Highlighting}[]
\FunctionTok{library}\NormalTok{(nlme)}
\DocumentationTok{\#\#\#\#PROCEDIMIENTO DE ZUUR PARA LA SELECCION DEL MODELO}
\DocumentationTok{\#\#1. Determinamos los efectos variables y para ello saturamos los efectos fijos}
\NormalTok{leon\_tb\_l}\OtherTok{\textless{}{-}}\NormalTok{leon\_tb\_l[}\SpecialCharTok{!}\FunctionTok{is.na}\NormalTok{(leon\_tb\_l}\SpecialCharTok{$}\NormalTok{Biomasa),]}
\NormalTok{m1a}\OtherTok{\textless{}{-}}\FunctionTok{gls}\NormalTok{(Biomasa}\SpecialCharTok{\textasciitilde{}}\DecValTok{1}\SpecialCharTok{+}\NormalTok{año,}\AttributeTok{data=}\NormalTok{leon\_tb\_l,}\AttributeTok{method=}\StringTok{"REML"}\NormalTok{)}
\NormalTok{m1b}\OtherTok{\textless{}{-}}\FunctionTok{lme}\NormalTok{(Biomasa}\SpecialCharTok{\textasciitilde{}}\DecValTok{1}\SpecialCharTok{+}\NormalTok{año,}\AttributeTok{random=}\SpecialCharTok{\textasciitilde{}}\DecValTok{1}\SpecialCharTok{|}\NormalTok{Estacion,}\AttributeTok{data=}\NormalTok{leon\_tb\_l,}\AttributeTok{method=}\StringTok{"REML"}\NormalTok{)}
\FunctionTok{anova}\NormalTok{(m1a,m1b)}\CommentTok{\#Efectivamente al año parece afectar a la VD, seleccionamos como variable aleatoria la estación.}
\end{Highlighting}
\end{Shaded}

\begin{verbatim}
##     Model df      AIC      BIC    logLik   Test  L.Ratio p-value
## m1a     1  9 416.4606 447.7864 -199.2303                        
## m1b     2 10 347.5091 382.3155 -163.7545 1 vs 2 70.95153  <.0001
\end{verbatim}

\begin{Shaded}
\begin{Highlighting}[]
\DocumentationTok{\#\#2. Seleccionamos las variables de la estructura fija.}
\NormalTok{m1c}\OtherTok{\textless{}{-}}\FunctionTok{lme}\NormalTok{(Biomasa}\SpecialCharTok{\textasciitilde{}}\DecValTok{1}\NormalTok{,}\AttributeTok{random=}\SpecialCharTok{\textasciitilde{}}\DecValTok{1}\SpecialCharTok{|}\NormalTok{Estacion,}\AttributeTok{data=}\NormalTok{leon\_tb\_l,}\AttributeTok{method=}\StringTok{"ML"}\NormalTok{)}
\NormalTok{m1d}\OtherTok{\textless{}{-}}\FunctionTok{lme}\NormalTok{(Biomasa}\SpecialCharTok{\textasciitilde{}}\DecValTok{1}\SpecialCharTok{+}\NormalTok{año,}\AttributeTok{random=}\SpecialCharTok{\textasciitilde{}}\DecValTok{1}\SpecialCharTok{|}\NormalTok{Estacion,}\AttributeTok{data=}\NormalTok{leon\_tb\_l,}\AttributeTok{method=}\StringTok{"ML"}\NormalTok{)}
\FunctionTok{anova}\NormalTok{(m1c,m1d)}\CommentTok{\#El modelo m1d es el elegido}
\end{Highlighting}
\end{Shaded}

\begin{verbatim}
##     Model df      AIC      BIC    logLik   Test L.Ratio p-value
## m1c     1  3 406.8656 417.4058 -200.4328                       
## m1d     2 10 323.3767 358.5110 -151.6883 1 vs 2 97.4889  <.0001
\end{verbatim}

\begin{Shaded}
\begin{Highlighting}[]
\DocumentationTok{\#\#3. Ajuste del modelo final con el método REML}
\NormalTok{m1d\_final}\OtherTok{\textless{}{-}}\FunctionTok{lme}\NormalTok{(Biomasa}\SpecialCharTok{\textasciitilde{}}\DecValTok{1}\SpecialCharTok{+}\NormalTok{año,}\AttributeTok{random=}\SpecialCharTok{\textasciitilde{}}\DecValTok{1}\SpecialCharTok{|}\NormalTok{Estacion,}\AttributeTok{method=}\StringTok{"REML"}\NormalTok{,}\AttributeTok{data=}\NormalTok{leon\_tb\_l)}
\FunctionTok{summary}\NormalTok{(m1d\_final)}\CommentTok{\#Diferencias entre 2017 y 2016 respecto de 2014}
\end{Highlighting}
\end{Shaded}

\begin{verbatim}
## Linear mixed-effects model fit by REML
##   Data: leon_tb_l 
##        AIC      BIC    logLik
##   347.5091 382.3155 -163.7545
## 
## Random effects:
##  Formula: ~1 | Estacion
##         (Intercept)  Residual
## StdDev:   0.3922224 0.3489627
## 
## Fixed effects:  Biomasa ~ 1 + año 
##                 Value  Std.Error  DF   t-value p-value
## (Intercept) 1.3458764 0.07720062 151 17.433492  0.0000
## año2015     0.1004071 0.11249997 151  0.892508  0.3735
## año2016     0.3313852 0.19989252 151  1.657817  0.0994
## año2017     0.5129883 0.07799500 151  6.577195  0.0000
## año2018     0.2711809 0.10599031 151  2.558545  0.0115
## año2019     0.6574566 0.18210715 151  3.610273  0.0004
## año2020     0.5550115 0.07737549 151  7.172963  0.0000
## año2021     0.7391176 0.10509690 151  7.032725  0.0000
##  Correlation: 
##         (Intr) añ2015 añ2016 añ2017 añ2018 añ2019 añ2020
## año2015 -0.613                                          
## año2016 -0.283  0.269                                   
## año2017 -0.531  0.379  0.225                            
## año2018 -0.654  0.665  0.286  0.403                     
## año2019 -0.315  0.302  0.239  0.250  0.319              
## año2020 -0.588  0.411  0.235  0.525  0.436  0.262       
## año2021 -0.661  0.670  0.287  0.406  0.716  0.320  0.440
## 
## Standardized Within-Group Residuals:
##           Min            Q1           Med            Q3           Max 
## -2.1725640501 -0.5476011726 -0.0002408912  0.5722837173  2.4257130817 
## 
## Number of Observations: 248
## Number of Groups: 90
\end{verbatim}

El modelo final lo utilizamos patra contrastar los diferentes niveles de
la variable predictora, en este caso los diferentes años mediante
comparaciones múltiples mediante eltest de Tukey. Comrobamos el modelo
de forma gráfica

\includegraphics{Informe_files/figure-latex/unnamed-chunk-17-1.pdf}
\includegraphics{Informe_files/figure-latex/unnamed-chunk-17-2.pdf}
\includegraphics{Informe_files/figure-latex/unnamed-chunk-17-3.pdf}
\includegraphics{Informe_files/figure-latex/unnamed-chunk-17-4.pdf}

\begin{verbatim}
## 
##   Simultaneous Tests for General Linear Hypotheses
## 
## Multiple Comparisons of Means: Tukey Contrasts
## 
## 
## Fit: lme.formula(fixed = Biomasa ~ 1 + año, data = leon_tb_l, random = ~1 | 
##     Estacion, method = "REML")
## 
## Linear Hypotheses:
##                  Estimate Std. Error z value Pr(>|z|)    
## 2015 - 2014 == 0  0.10041    0.11250   0.893  0.98401    
## 2016 - 2014 == 0  0.33139    0.19989   1.658  0.68271    
## 2017 - 2014 == 0  0.51299    0.07799   6.577  < 0.001 ***
## 2018 - 2014 == 0  0.27118    0.10599   2.559  0.14942    
## 2019 - 2014 == 0  0.65746    0.18211   3.610  0.00609 ** 
## 2020 - 2014 == 0  0.55501    0.07738   7.173  < 0.001 ***
## 2021 - 2014 == 0  0.73912    0.10510   7.033  < 0.001 ***
## 2016 - 2015 == 0  0.23098    0.20123   1.148  0.93642    
## 2017 - 2015 == 0  0.41258    0.10992   3.753  0.00360 ** 
## 2018 - 2015 == 0  0.17077    0.08960   1.906  0.50961    
## 2019 - 2015 == 0  0.55705    0.18283   3.047  0.04028 *  
## 2020 - 2015 == 0  0.45460    0.10721   4.240  < 0.001 ***
## 2021 - 2015 == 0  0.63871    0.08860   7.209  < 0.001 ***
## 2017 - 2016 == 0  0.18160    0.19758   0.919  0.98105    
## 2018 - 2016 == 0 -0.06020    0.19766  -0.305  0.99999    
## 2019 - 2016 == 0  0.32607    0.23599   1.382  0.84593    
## 2020 - 2016 == 0  0.22363    0.19663   1.137  0.93948    
## 2021 - 2016 == 0  0.40773    0.19735   2.066  0.40059    
## 2018 - 2017 == 0 -0.24181    0.10322  -2.343  0.24096    
## 2019 - 2017 == 0  0.14447    0.17931   0.806  0.99128    
## 2020 - 2017 == 0  0.04202    0.07575   0.555  0.99917    
## 2021 - 2017 == 0  0.22613    0.10233   2.210  0.31193    
## 2019 - 2018 == 0  0.38628    0.17912   2.157  0.34358    
## 2020 - 2018 == 0  0.28383    0.10031   2.829  0.07493 .  
## 2021 - 2018 == 0  0.46794    0.07951   5.885  < 0.001 ***
## 2020 - 2019 == 0 -0.10245    0.17825  -0.575  0.99896    
## 2021 - 2019 == 0  0.08166    0.17877   0.457  0.99977    
## 2021 - 2020 == 0  0.18411    0.09938   1.853  0.54730    
## ---
## Signif. codes:  0 '***' 0.001 '**' 0.01 '*' 0.05 '.' 0.1 ' ' 1
## (Adjusted p values reported -- single-step method)
\end{verbatim}

Podemos ver en la tabla de contrastes que para los datos de biomasa
existen diferencias significativas de los datos del año 2014 respecto a
los de 2017 en adelante, si exceptuamos el año 2018, año que parece
producirse una reducción en la biomasa.

\hypertarget{anuxe1lisis-de-la-variable-densidad}{%
\section{2. ANÁLISIS DE LA VARIABLE
DENSIDAD}\label{anuxe1lisis-de-la-variable-densidad}}

Analizaremos la siguiente variable poblacional que corresponde con
densidad de trucha en ind/m2. Como en el caso de la biomasa, interesa
conocer la existencia de diferencias significativas derivadas de los
diferentes inventarios para ver si el cambio de políticas de gestión que
se produce en el año 2013 y que tiene como primer año de implementación
el 2014, ha supuesto un cambio en las variables de las poblaciones de
trucha. Como en el caso de la biomasa, se escogieron los libres por ser
los que más se notaría las medidas implementadas al declararse como de
pesca sin muerte. Observamos en primer lugar la distribución de las
variables poblacionales, como variables respuesta, respecto a los grupos
anuales y a los sistemas de gestión.

\hypertarget{visualizaciuxf3n-de-los-datos-1}{%
\subsection{2.1 VISUALIZACIÓN DE LOS
DATOS}\label{visualizaciuxf3n-de-los-datos-1}}

\begin{verbatim}
##     Estacion   Provincia  Gestion Nivel        año         Biomasa      
##  LE004  :  8   Leon:720   L:720   1: 40   2014   : 90   Min.   : 0.010  
##  LE009  :  8                      2:680   2015   : 90   1st Qu.: 2.627  
##  LE010  :  8                              2016   : 90   Median : 4.925  
##  LE070  :  8                              2017   : 90   Mean   : 6.802  
##  LE072  :  8                              2018   : 90   3rd Qu.: 9.750  
##  LE073  :  8                              2019   : 90   Max.   :38.120  
##  (Other):672                              (Other):180   NA's   :472     
##     Densidad        Peso_medio   
##  Min.   :0.0000   Min.   : 4.50  
##  1st Qu.:0.0600   1st Qu.:21.82  
##  Median :0.1400   Median :37.00  
##  Mean   :0.2054   Mean   :  Inf  
##  3rd Qu.:0.2650   3rd Qu.:54.82  
##  Max.   :2.0400   Max.   :  Inf  
##  NA's   :473      NA's   :473
\end{verbatim}

\includegraphics{Informe_files/figure-latex/unnamed-chunk-19-1.pdf}

Representamos las variables para los tramos libres, donde es previsible
que se puedan observar con los efectos que se quieren comprobar.

\includegraphics{Informe_files/figure-latex/unnamed-chunk-20-1.pdf}

Se observa la existencia de posibles outliers que para la comprobación
de las comparaciones y realizar las comprobaciones de hipótesis deberá
ser tenida en cuenta para elegir la tecnica estadística. Determinamos a
continuación los valores medios para cada año para los tramos de pesca
libre, recogiéndose igualmente las desviaciones estandar de cada grupo.

\hypertarget{medias-y-desviaciones-de-la-variable-densidad}{%
\subsection{2.2 MEDIAS Y DESVIACIONES DE LA VARIABLE
DENSIDAD}\label{medias-y-desviaciones-de-la-variable-densidad}}

Para los tramos libres de León, comprobamos los valores medias de
biomasa de trucha (en ind/m2) en los diferentes años, así como los
valores de la desviación estándar.

\begin{longtable}[]{@{}
  >{\raggedright\arraybackslash}p{(\columnwidth - 16\tabcolsep) * \real{0.0698}}
  >{\raggedleft\arraybackslash}p{(\columnwidth - 16\tabcolsep) * \real{0.1163}}
  >{\raggedleft\arraybackslash}p{(\columnwidth - 16\tabcolsep) * \real{0.1163}}
  >{\raggedleft\arraybackslash}p{(\columnwidth - 16\tabcolsep) * \real{0.1163}}
  >{\raggedleft\arraybackslash}p{(\columnwidth - 16\tabcolsep) * \real{0.1163}}
  >{\raggedleft\arraybackslash}p{(\columnwidth - 16\tabcolsep) * \real{0.1163}}
  >{\raggedleft\arraybackslash}p{(\columnwidth - 16\tabcolsep) * \real{0.1163}}
  >{\raggedleft\arraybackslash}p{(\columnwidth - 16\tabcolsep) * \real{0.1163}}
  >{\raggedleft\arraybackslash}p{(\columnwidth - 16\tabcolsep) * \real{0.1163}}@{}}
\caption{Valores medios y desviaciones estandar}\tabularnewline
\toprule\noalign{}
\begin{minipage}[b]{\linewidth}\raggedright
\end{minipage} & \begin{minipage}[b]{\linewidth}\raggedleft
2014
\end{minipage} & \begin{minipage}[b]{\linewidth}\raggedleft
2015
\end{minipage} & \begin{minipage}[b]{\linewidth}\raggedleft
2016
\end{minipage} & \begin{minipage}[b]{\linewidth}\raggedleft
2017
\end{minipage} & \begin{minipage}[b]{\linewidth}\raggedleft
2018
\end{minipage} & \begin{minipage}[b]{\linewidth}\raggedleft
2019
\end{minipage} & \begin{minipage}[b]{\linewidth}\raggedleft
2020
\end{minipage} & \begin{minipage}[b]{\linewidth}\raggedleft
2021
\end{minipage} \\
\midrule\noalign{}
\endfirsthead
\toprule\noalign{}
\begin{minipage}[b]{\linewidth}\raggedright
\end{minipage} & \begin{minipage}[b]{\linewidth}\raggedleft
2014
\end{minipage} & \begin{minipage}[b]{\linewidth}\raggedleft
2015
\end{minipage} & \begin{minipage}[b]{\linewidth}\raggedleft
2016
\end{minipage} & \begin{minipage}[b]{\linewidth}\raggedleft
2017
\end{minipage} & \begin{minipage}[b]{\linewidth}\raggedleft
2018
\end{minipage} & \begin{minipage}[b]{\linewidth}\raggedleft
2019
\end{minipage} & \begin{minipage}[b]{\linewidth}\raggedleft
2020
\end{minipage} & \begin{minipage}[b]{\linewidth}\raggedleft
2021
\end{minipage} \\
\midrule\noalign{}
\endhead
\bottomrule\noalign{}
\endlastfoot
Media & 0.0807500 & 0.1100000 & 0.2450000 & 0.2319512 & 0.1360526 &
0.3725000 & 0.2425490 & 0.3215000 \\
Desv & 0.0860944 & 0.1228971 & 0.2655184 & 0.1832378 & 0.1073146 &
0.2718302 & 0.1828753 & 0.2326828 \\
\end{longtable}

Se observa un aumento de los valores medios respecto a 2014.
Representamos la distribución mediante histograma de la variable
densidad

\includegraphics{Informe_files/figure-latex/unnamed-chunk-22-1.pdf}

\hypertarget{modelo-para-la-comparaciuxf3n-de-la-densidad}{%
\subsection{2.3 MODELO PARA LA COMPARACIÓN DE LA
DENSIDAD}\label{modelo-para-la-comparaciuxf3n-de-la-densidad}}

Para la comparación de los grupos optamos en primer lugar por observar
los datos y ver los supuestos para adaptarlo a un modelo lineal.
Comprobamos la normalidad y la homocedasticidad. Primero los
visualizamos

\includegraphics{Informe_files/figure-latex/unnamed-chunk-23-1.pdf}
\includegraphics{Informe_files/figure-latex/unnamed-chunk-23-2.pdf}

Comprobamos la normalidad y homocedasticidad aunque es visible que no
parecen cumplirse ninguno de los 2 criterios

\begin{verbatim}
## 
##  Shapiro-Wilk normality test
## 
## data:  leon_l$Densidad
## W = 0.83854, p-value = 2.69e-15
\end{verbatim}

No se puede aceptar la la distribución normal de los datos

\begin{verbatim}
## Levene's Test for Homogeneity of Variance (center = median)
##        Df F value    Pr(>F)    
## group   7  5.2825 1.259e-05 ***
##       238                      
## ---
## Signif. codes:  0 '***' 0.001 '**' 0.01 '*' 0.05 '.' 0.1 ' ' 1
\end{verbatim}

Tampoco se puede aceptar la homocedasticidad de los datos de densidad
para los diferentes años. Por ello es preciso transformar los datos y
ver si con ello podemos asumir un modelo lineal. Para ello
determinaremos los valores de \(\lambda\) mediante la aplicación de una
transformación de Box Cox

\includegraphics{Informe_files/figure-latex/unnamed-chunk-26-1.pdf}

El valor de \(\lambda\) obtenido es 0.1818182 es el valor de la potencia
usada en la transformación de los datos.

Una vez transformado los datos visualizamos nuevamente
\includegraphics{Informe_files/figure-latex/unnamed-chunk-28-1.pdf}
\includegraphics{Informe_files/figure-latex/unnamed-chunk-28-2.pdf}

Visualmente ahora parece que los datos pueden cumplir alos criterios de
normalidad y homocedasticidad, pese a ello realizamos los test.

\begin{verbatim}
## 
##  Shapiro-Wilk normality test
## 
## data:  leon_tb_l$Densidad
## W = 0.98914, p-value = 0.06927
\end{verbatim}

Se acepta la la distribución normal de los datos

\begin{verbatim}
## Levene's Test for Homogeneity of Variance (center = median)
##        Df F value Pr(>F)
## group   7  0.6972 0.6744
##       231
\end{verbatim}

Se acepta la homocedasticidad

Planteamos ahora un modelo lineal mixto. Esto es conveniente en nuestro
caso ya que podemos establecer una parte aleatoria del modelo con los
datos anidados por la variable Estación. Como término fijo utilizaríamos
la variable año, ya que se trata del factor principal que nos explica
precisamente si existen diferencias en los valores de la densidad
considerando el año 2014 como el año que se produce el cambio de
política de gestión al establecer los libres como libres sin muerte.
Para la construcción del modelo seguimos el procedimiento que recomienda
Zuur et al.~(2009) y que se secuenciaría en fases:

1.Estructura aleatoria óptima. Usando un modelo saturado (beyond optimal
model), se determina la estructura óptima del componente aleatorio, la
cual no debe contener información que esté en la componente fija.
Debemos:

\begin{itemize}
\item
  construir un modelo saturado.
\item
  comparar modelos con distinta estructura aleatoria, mediante máxima
  verosimilitud restringida (REML).
\end{itemize}

2.Estructura fija óptima. Una vez encontramos la estructura aleatoria
óptima, podemos encontrar la estructura fija óptima. Comparamos los
modelos anidados mediante máxima verosimilitud (ML), manteniendo la
misma estructura aleatoria.

3.Ajuste del modelo final con REML.

\begin{Shaded}
\begin{Highlighting}[]
\FunctionTok{library}\NormalTok{(nlme)}
\DocumentationTok{\#\#\#\#PROCEDIMIENTO DE ZUUR PARA LA SELECCION DEL MODELO}
\DocumentationTok{\#\#1. Determinamos los efectos variables y para ello saturamos los efectos fijos}
\NormalTok{leon\_tb\_l}\OtherTok{\textless{}{-}}\NormalTok{leon\_tb\_l[}\SpecialCharTok{!}\FunctionTok{is.na}\NormalTok{(leon\_tb\_l}\SpecialCharTok{$}\NormalTok{Densidad),]}
\NormalTok{m1a}\OtherTok{\textless{}{-}}\FunctionTok{gls}\NormalTok{(Densidad}\SpecialCharTok{\textasciitilde{}}\DecValTok{1}\SpecialCharTok{+}\NormalTok{año,}\AttributeTok{data=}\NormalTok{leon\_tb\_l,}\AttributeTok{method=}\StringTok{"REML"}\NormalTok{)}
\NormalTok{m1b}\OtherTok{\textless{}{-}}\FunctionTok{lme}\NormalTok{(Densidad}\SpecialCharTok{\textasciitilde{}}\DecValTok{1}\SpecialCharTok{+}\NormalTok{año,}\AttributeTok{random=}\SpecialCharTok{\textasciitilde{}}\DecValTok{1}\SpecialCharTok{|}\NormalTok{Estacion,}\AttributeTok{data=}\NormalTok{leon\_tb\_l,}\AttributeTok{method=}\StringTok{"REML"}\NormalTok{)}
\FunctionTok{anova}\NormalTok{(m1a,m1b)}\CommentTok{\#Efectivamente al año parece afectar a la VD, seleccionamos como variable aleatoria la estación.}
\end{Highlighting}
\end{Shaded}

\begin{verbatim}
##     Model df       AIC       BIC   logLik   Test  L.Ratio p-value
## m1a     1  9 -321.1165 -290.1348 169.5583                        
## m1b     2 10 -362.0582 -327.6340 191.0291 1 vs 2 42.94169  <.0001
\end{verbatim}

\begin{Shaded}
\begin{Highlighting}[]
\DocumentationTok{\#\#2. Seleccionamos las variables de la estructura fija.}
\NormalTok{m1c}\OtherTok{\textless{}{-}}\FunctionTok{lme}\NormalTok{(Densidad}\SpecialCharTok{\textasciitilde{}}\DecValTok{1}\NormalTok{,}\AttributeTok{random=}\SpecialCharTok{\textasciitilde{}}\DecValTok{1}\SpecialCharTok{|}\NormalTok{Estacion,}\AttributeTok{data=}\NormalTok{leon\_tb\_l,}\AttributeTok{method=}\StringTok{"ML"}\NormalTok{)}
\NormalTok{m1d}\OtherTok{\textless{}{-}}\FunctionTok{lme}\NormalTok{(Densidad}\SpecialCharTok{\textasciitilde{}}\DecValTok{1}\SpecialCharTok{+}\NormalTok{año,}\AttributeTok{random=}\SpecialCharTok{\textasciitilde{}}\DecValTok{1}\SpecialCharTok{|}\NormalTok{Estacion,}\AttributeTok{data=}\NormalTok{leon\_tb\_l,}\AttributeTok{method=}\StringTok{"ML"}\NormalTok{)}
\FunctionTok{anova}\NormalTok{(m1c,m1d)}\CommentTok{\#El modelo m1d es el elegido}
\end{Highlighting}
\end{Shaded}

\begin{verbatim}
##     Model df       AIC       BIC   logLik   Test  L.Ratio p-value
## m1c     1  3 -328.2504 -317.8210 167.1252                        
## m1d     2 10 -409.7011 -374.9365 214.8506 1 vs 2 95.45076  <.0001
\end{verbatim}

\begin{Shaded}
\begin{Highlighting}[]
\DocumentationTok{\#\#3. Ajuste del modelo final con el método REML}
\NormalTok{m1d\_final}\OtherTok{\textless{}{-}}\FunctionTok{lme}\NormalTok{(Densidad}\SpecialCharTok{\textasciitilde{}}\DecValTok{1}\SpecialCharTok{+}\NormalTok{año,}\AttributeTok{random=}\SpecialCharTok{\textasciitilde{}}\DecValTok{1}\SpecialCharTok{|}\NormalTok{Estacion,}\AttributeTok{method=}\StringTok{"REML"}\NormalTok{,}\AttributeTok{data=}\NormalTok{leon\_tb\_l)}
\FunctionTok{summary}\NormalTok{(m1d\_final)}\CommentTok{\#Diferencias entre 2017 y 2016 respecto de 2014}
\end{Highlighting}
\end{Shaded}

\begin{verbatim}
## Linear mixed-effects model fit by REML
##   Data: leon_tb_l 
##         AIC      BIC   logLik
##   -362.0582 -327.634 191.0291
## 
## Random effects:
##  Formula: ~1 | Estacion
##         (Intercept)   Residual
## StdDev:  0.07692056 0.08021259
## 
## Fixed effects:  Densidad ~ 1 + año 
##                 Value  Std.Error  DF  t-value p-value
## (Intercept) 0.5892415 0.01722955 143 34.19947  0.0000
## año2015     0.0529923 0.02553027 143  2.07567  0.0397
## año2016     0.0804010 0.04584639 143  1.75370  0.0816
## año2017     0.1312110 0.01836909 143  7.14303  0.0000
## año2018     0.0786468 0.02369991 143  3.31844  0.0011
## año2019     0.1659635 0.04584639 143  3.61999  0.0004
## año2020     0.1471381 0.01811142 143  8.12405  0.0000
## año2021     0.1861422 0.02347420 143  7.92965  0.0000
##  Correlation: 
##         (Intr) añ2015 añ2016 añ2017 añ2018 añ2019 añ2020
## año2015 -0.622                                          
## año2016 -0.294  0.261                                   
## año2017 -0.575  0.400  0.232                            
## año2018 -0.675  0.631  0.280  0.431                     
## año2019 -0.294  0.261  0.235  0.232  0.280              
## año2020 -0.637  0.434  0.244  0.541  0.469  0.244       
## año2021 -0.683  0.637  0.281  0.435  0.688  0.281  0.473
## 
## Standardized Within-Group Residuals:
##        Min         Q1        Med         Q3        Max 
## -2.5724777 -0.5170112 -0.0324419  0.6336792  2.2712043 
## 
## Number of Observations: 239
## Number of Groups: 89
\end{verbatim}

El modelo final lo utilizamos patra contrastar los diferentes niveles de
la variable predictora, en este caso los diferentes años mediante
comparaciones múltiples mediante eltest de Tukey. Comrobamos el modelo
de forma gráfica

\includegraphics{Informe_files/figure-latex/unnamed-chunk-32-1.pdf}
\includegraphics{Informe_files/figure-latex/unnamed-chunk-32-2.pdf}
\includegraphics{Informe_files/figure-latex/unnamed-chunk-32-3.pdf}
\includegraphics{Informe_files/figure-latex/unnamed-chunk-32-4.pdf}

\begin{verbatim}
## 
##   Simultaneous Tests for General Linear Hypotheses
## 
## Multiple Comparisons of Means: Tukey Contrasts
## 
## 
## Fit: lme.formula(fixed = Densidad ~ 1 + año, data = leon_tb_l, random = ~1 | 
##     Estacion, method = "REML")
## 
## Linear Hypotheses:
##                   Estimate Std. Error z value Pr(>|z|)    
## 2015 - 2014 == 0  0.052992   0.025530   2.076  0.39491    
## 2016 - 2014 == 0  0.080401   0.045846   1.754  0.61746    
## 2017 - 2014 == 0  0.131211   0.018369   7.143  < 0.001 ***
## 2018 - 2014 == 0  0.078647   0.023700   3.318  0.01696 *  
## 2019 - 2014 == 0  0.165963   0.045846   3.620  0.00586 ** 
## 2020 - 2014 == 0  0.147138   0.018111   8.124  < 0.001 ***
## 2021 - 2014 == 0  0.186142   0.023474   7.930  < 0.001 ***
## 2016 - 2015 == 0  0.027409   0.046280   0.592  0.99874    
## 2017 - 2015 == 0  0.078219   0.024782   3.156  0.02882 *  
## 2018 - 2015 == 0  0.025654   0.021198   1.210  0.91732    
## 2019 - 2015 == 0  0.112971   0.046280   2.441  0.19553    
## 2020 - 2015 == 0  0.094146   0.024043   3.916  0.00193 ** 
## 2021 - 2015 == 0  0.133150   0.020954   6.354  < 0.001 ***
## 2017 - 2016 == 0  0.050810   0.045254   1.123  0.94345    
## 2018 - 2016 == 0 -0.001754   0.045325  -0.039  1.00000    
## 2019 - 2016 == 0  0.085562   0.056719   1.509  0.77783    
## 2020 - 2016 == 0  0.066737   0.044988   1.483  0.79234    
## 2021 - 2016 == 0  0.105741   0.045245   2.337  0.24391    
## 2018 - 2017 == 0 -0.052564   0.022891  -2.296  0.26476    
## 2019 - 2017 == 0  0.034752   0.045254   0.768  0.99352    
## 2020 - 2017 == 0  0.015927   0.017484   0.911  0.98208    
## 2021 - 2017 == 0  0.054931   0.022663   2.424  0.20306    
## 2019 - 2018 == 0  0.087317   0.045325   1.926  0.49579    
## 2020 - 2018 == 0  0.068491   0.022078   3.102  0.03401 *  
## 2021 - 2018 == 0  0.107495   0.018624   5.772  < 0.001 ***
## 2020 - 2019 == 0 -0.018825   0.044988  -0.418  0.99987    
## 2021 - 2019 == 0  0.020179   0.045245   0.446  0.99980    
## 2021 - 2020 == 0  0.039004   0.021838   1.786  0.59456    
## ---
## Signif. codes:  0 '***' 0.001 '**' 0.01 '*' 0.05 '.' 0.1 ' ' 1
## (Adjusted p values reported -- single-step method)
\end{verbatim}

Podemos ver en la tabla de contrastes que para los datos de biomasa
existen diferencias significativas de los datos del año 2014 respecto a
los de 2017 en adelante.

\hypertarget{anuxe1lisis-de-la-variable-peso-medio}{%
\section{3. ANÁLISIS DE LA VARIABLE PESO
MEDIO}\label{anuxe1lisis-de-la-variable-peso-medio}}

Otra variable determinada de las 2 anteriores por simple cocientre entre
la biomasa y la densidad es el peso medio (gr/ind). Con ella podemos
comprobar si desde 2014 ha habido variaciones significativas en la
misma, también centrándonos en los tramos libres, desde la entrada de la
normativa que exige la pesca sin muerte.

\hypertarget{visualizaciuxf3n-de-los-datos-2}{%
\subsection{3.1 VISUALIZACIÓN DE LOS
DATOS}\label{visualizaciuxf3n-de-los-datos-2}}

\begin{verbatim}
##     Estacion   Provincia  Gestion Nivel        año        Biomasa      
##  LE009  :  8   Leon:245   L:245   1: 34   2020   :51   Min.   : 0.010  
##  LE010  :  8                      2:211   2017   :41   1st Qu.: 2.630  
##  LE004  :  7                              2021   :40   Median : 4.930  
##  LE070  :  6                              2014   :39   Mean   : 6.791  
##  LE189  :  5                              2018   :38   3rd Qu.: 9.740  
##  LE072  :  3                              2015   :28   Max.   :38.120  
##  (Other):208                              (Other): 8                   
##     Densidad        Peso_medio    
##  Min.   :0.0000   Min.   :  4.50  
##  1st Qu.:0.0600   1st Qu.: 22.04  
##  Median :0.1400   Median : 37.00  
##  Mean   :0.1987   Mean   : 45.90  
##  3rd Qu.:0.2600   3rd Qu.: 54.81  
##  Max.   :0.9400   Max.   :382.00  
## 
\end{verbatim}

\includegraphics{Informe_files/figure-latex/unnamed-chunk-34-1.pdf}

Representamos las variables para los tramos libres, donde es previsible
que se puedan observar con los efectos que se quieren comprobar por la
introducción de la modalidad de sin muerte de forma general.

\includegraphics{Informe_files/figure-latex/unnamed-chunk-35-1.pdf}

Se observa la existencia de outliers que para la comprobación de las
comparaciones y realizar las comprobaciones de hipótesis deberá ser
tenida en cuenta para elegir la tecnica estadística. Determinamos a
continuación los valores medios para cada año para los tramos de pesca
libre, recogiéndose igualmente las desviaciones estandar de cada grupo.

\hypertarget{medias-y-desviaciones-de-la-variable-peso-medio}{%
\subsection{3.2 MEDIAS Y DESVIACIONES DE LA VARIABLE PESO
MEDIO}\label{medias-y-desviaciones-de-la-variable-peso-medio}}

Para los tramos libres de León, comprobamos los valores medios del peso
medio de trucha (en gr/ind) en los diferentes años, así como los valores
de la desviación estándar.

\begin{longtable}[]{@{}
  >{\raggedright\arraybackslash}p{(\columnwidth - 16\tabcolsep) * \real{0.0769}}
  >{\raggedleft\arraybackslash}p{(\columnwidth - 16\tabcolsep) * \real{0.1154}}
  >{\raggedleft\arraybackslash}p{(\columnwidth - 16\tabcolsep) * \real{0.1154}}
  >{\raggedleft\arraybackslash}p{(\columnwidth - 16\tabcolsep) * \real{0.1154}}
  >{\raggedleft\arraybackslash}p{(\columnwidth - 16\tabcolsep) * \real{0.1154}}
  >{\raggedleft\arraybackslash}p{(\columnwidth - 16\tabcolsep) * \real{0.1154}}
  >{\raggedleft\arraybackslash}p{(\columnwidth - 16\tabcolsep) * \real{0.1154}}
  >{\raggedleft\arraybackslash}p{(\columnwidth - 16\tabcolsep) * \real{0.1154}}
  >{\raggedleft\arraybackslash}p{(\columnwidth - 16\tabcolsep) * \real{0.1154}}@{}}
\caption{Valores medios y desviaciones estandar}\tabularnewline
\toprule\noalign{}
\begin{minipage}[b]{\linewidth}\raggedright
\end{minipage} & \begin{minipage}[b]{\linewidth}\raggedleft
2014
\end{minipage} & \begin{minipage}[b]{\linewidth}\raggedleft
2015
\end{minipage} & \begin{minipage}[b]{\linewidth}\raggedleft
2016
\end{minipage} & \begin{minipage}[b]{\linewidth}\raggedleft
2017
\end{minipage} & \begin{minipage}[b]{\linewidth}\raggedleft
2018
\end{minipage} & \begin{minipage}[b]{\linewidth}\raggedleft
2019
\end{minipage} & \begin{minipage}[b]{\linewidth}\raggedleft
2020
\end{minipage} & \begin{minipage}[b]{\linewidth}\raggedleft
2021
\end{minipage} \\
\midrule\noalign{}
\endfirsthead
\toprule\noalign{}
\begin{minipage}[b]{\linewidth}\raggedright
\end{minipage} & \begin{minipage}[b]{\linewidth}\raggedleft
2014
\end{minipage} & \begin{minipage}[b]{\linewidth}\raggedleft
2015
\end{minipage} & \begin{minipage}[b]{\linewidth}\raggedleft
2016
\end{minipage} & \begin{minipage}[b]{\linewidth}\raggedleft
2017
\end{minipage} & \begin{minipage}[b]{\linewidth}\raggedleft
2018
\end{minipage} & \begin{minipage}[b]{\linewidth}\raggedleft
2019
\end{minipage} & \begin{minipage}[b]{\linewidth}\raggedleft
2020
\end{minipage} & \begin{minipage}[b]{\linewidth}\raggedleft
2021
\end{minipage} \\
\midrule\noalign{}
\endhead
\bottomrule\noalign{}
\endlastfoot
Media & 60.63667 & 40.66071 & 48.17750 & 44.06390 & 42.72500 & 34.93000
& 41.23627 & 46.91050 \\
Desv & 37.93280 & 33.46672 & 29.31809 & 35.79717 & 26.25097 & 13.07535 &
26.50332 & 62.37319 \\
\end{longtable}

Se observa un aumento de los valores medios respecto a 2014.
Representamos la distribución mediante histograma de la variable
densidad

\includegraphics{Informe_files/figure-latex/unnamed-chunk-37-1.pdf}

\hypertarget{modelo-para-la-comparaciuxf3n-del-peso-medio}{%
\subsection{2.3 MODELO PARA LA COMPARACIÓN DEL PESO
MEDIO}\label{modelo-para-la-comparaciuxf3n-del-peso-medio}}

Para la comparación de los grupos optamos en primer lugar por observar
los datos y ver los supuestos para adaptarlo a un modelo lineal.
Comprobamos la normalidad y la homocedasticidad. Primero los
visualizamos

\includegraphics{Informe_files/figure-latex/unnamed-chunk-38-1.pdf}
\includegraphics{Informe_files/figure-latex/unnamed-chunk-38-2.pdf}

Comprobamos la normalidad y homocedasticidad aunque es visible que no
parecen cumplirse ninguno de los 2 criterios

\begin{verbatim}
## 
##  Shapiro-Wilk normality test
## 
## data:  leon_l$Densidad
## W = 0.83839, p-value = 2.86e-15
\end{verbatim}

No se puede aceptar la la distribución normal de los datos

\begin{verbatim}
## Levene's Test for Homogeneity of Variance (center = median)
##        Df F value    Pr(>F)    
## group   7  5.2073 1.541e-05 ***
##       237                      
## ---
## Signif. codes:  0 '***' 0.001 '**' 0.01 '*' 0.05 '.' 0.1 ' ' 1
\end{verbatim}

Tampoco se puede aceptar la homocedasticidad de los datos de densidad
para los diferentes años. Por ello es preciso transformar los datos y
ver si con ello podemos asumir un modelo lineal. Para ello
determinaremos los valores de \(\lambda\) mediante la aplicación de una
transformación de Box Cox

\includegraphics{Informe_files/figure-latex/unnamed-chunk-41-1.pdf}

El valor de \(\lambda\) obtenido es 0.020202 es el valor de la potencia
usada en la transformación de los datos.

Transformamos los datos

Una vez transformado los datos visualizamos nuevamente
\includegraphics{Informe_files/figure-latex/unnamed-chunk-43-1.pdf}
\includegraphics{Informe_files/figure-latex/unnamed-chunk-43-2.pdf}
Visualmente ahora parece que los datos pueden cumplir alos criterios de
normalidad y homocedasticidad, pese a ello realizamos los test.

\begin{verbatim}
## 
##  Shapiro-Wilk normality test
## 
## data:  leon_tb_l$Peso_medio
## W = 0.9941, p-value = 0.4743
\end{verbatim}

Se acepta la la distribución normal de los datos

\begin{verbatim}
## Levene's Test for Homogeneity of Variance (center = median)
##        Df F value Pr(>F)
## group   7  0.5712  0.779
##       231
\end{verbatim}

Se acepta la homocedasticidad

Planteamos ahora un modelo lineal mixto. Esto es conveniente en nuestro
caso ya que podemos establecer una parte aleatoria del modelo con los
datos anidados por la variable Estación. Como término fijo utilizaríamos
la variable año, ya que se trata del factor principal que nos explica
precisamente si existen diferencias en los valores de la densidad
considerando el año 2014 como el año que se produce el cambio de
política de gestión al establecer los libres como libres sin muerte.
Para la construcción del modelo seguimos el procedimiento que recomienda
Zuur et al.~(2009) y que se secuenciaría en fases: 1.Estructura
aleatoria óptima. Usando un modelo saturado (beyond optimal model), se
determina la estructura óptima del componente aleatorio, la cual no debe
contener información que esté en la componente fija. Debemos:

\begin{itemize}
\tightlist
\item
  construir un modelo saturado.
\item
  comparar modelos con distinta estructura aleatoria, mediante máxima
  verosimilitud restringida (REML).
\end{itemize}

2.Estructura fija óptima. Una vez encontramos la estructura aleatoria
óptima, podemos encontrar la estructura fija óptima. Comparamos los
modelos anidados mediante máxima verosimilitud (ML), manteniendo la
misma estructura aleatoria.

3.Ajuste del modelo final con REML.

\begin{Shaded}
\begin{Highlighting}[]
\FunctionTok{library}\NormalTok{(nlme)}
\DocumentationTok{\#\#\#\#PROCEDIMIENTO DE ZUUR PARA LA SELECCION DEL MODELO}
\DocumentationTok{\#\#1. Determinamos los efectos variables y para ello saturamos los efectos fijos}
\NormalTok{leon\_tb\_l}\OtherTok{\textless{}{-}}\NormalTok{leon\_tb\_l[}\SpecialCharTok{!}\FunctionTok{is.na}\NormalTok{(leon\_tb\_l}\SpecialCharTok{$}\NormalTok{Peso\_medio),]}
\NormalTok{m1a}\OtherTok{\textless{}{-}}\FunctionTok{gls}\NormalTok{(Peso\_medio}\SpecialCharTok{\textasciitilde{}}\DecValTok{1}\SpecialCharTok{+}\NormalTok{año,}\AttributeTok{data=}\NormalTok{leon\_tb\_l,}\AttributeTok{method=}\StringTok{"REML"}\NormalTok{)}
\NormalTok{m1b}\OtherTok{\textless{}{-}}\FunctionTok{lme}\NormalTok{(Peso\_medio}\SpecialCharTok{\textasciitilde{}}\DecValTok{1}\SpecialCharTok{+}\NormalTok{año,}\AttributeTok{random=}\SpecialCharTok{\textasciitilde{}}\DecValTok{1}\SpecialCharTok{|}\NormalTok{Estacion,}\AttributeTok{data=}\NormalTok{leon\_tb\_l,}\AttributeTok{method=}\StringTok{"REML"}\NormalTok{)}
\FunctionTok{anova}\NormalTok{(m1a,m1b)}\CommentTok{\#Efectivamente al año parece afectar a la VD, seleccionamos como variable aleatoria la estación.}
\end{Highlighting}
\end{Shaded}

\begin{verbatim}
##     Model df       AIC       BIC   logLik   Test  L.Ratio p-value
## m1a     1  9 -1250.487 -1219.505 634.2435                        
## m1b     2 10 -1286.370 -1251.946 653.1851 1 vs 2 37.88316  <.0001
\end{verbatim}

\begin{Shaded}
\begin{Highlighting}[]
\DocumentationTok{\#\#2. Seleccionamos las variables de la estructura fija.}
\NormalTok{m1c}\OtherTok{\textless{}{-}}\FunctionTok{lme}\NormalTok{(Peso\_medio}\SpecialCharTok{\textasciitilde{}}\DecValTok{1}\NormalTok{,}\AttributeTok{random=}\SpecialCharTok{\textasciitilde{}}\DecValTok{1}\SpecialCharTok{|}\NormalTok{Estacion,}\AttributeTok{data=}\NormalTok{leon\_tb\_l,}\AttributeTok{method=}\StringTok{"ML"}\NormalTok{)}
\NormalTok{m1d}\OtherTok{\textless{}{-}}\FunctionTok{lme}\NormalTok{(Peso\_medio}\SpecialCharTok{\textasciitilde{}}\DecValTok{1}\SpecialCharTok{+}\NormalTok{año,}\AttributeTok{random=}\SpecialCharTok{\textasciitilde{}}\DecValTok{1}\SpecialCharTok{|}\NormalTok{Estacion,}\AttributeTok{data=}\NormalTok{leon\_tb\_l,}\AttributeTok{method=}\StringTok{"ML"}\NormalTok{)}
\FunctionTok{anova}\NormalTok{(m1c,m1d)}\CommentTok{\#El modelo m1d es el elegido}
\end{Highlighting}
\end{Shaded}

\begin{verbatim}
##     Model df       AIC       BIC   logLik   Test  L.Ratio p-value
## m1c     1  3 -1363.442 -1353.013 684.7212                        
## m1d     2 10 -1366.083 -1331.319 693.0417 1 vs 2 16.64103  0.0199
\end{verbatim}

\begin{Shaded}
\begin{Highlighting}[]
\DocumentationTok{\#\#3. Ajuste del modelo final con el método REML}
\NormalTok{m1d\_final}\OtherTok{\textless{}{-}}\FunctionTok{lme}\NormalTok{(Peso\_medio}\SpecialCharTok{\textasciitilde{}}\DecValTok{1}\SpecialCharTok{+}\NormalTok{año,}\AttributeTok{random=}\SpecialCharTok{\textasciitilde{}}\DecValTok{1}\SpecialCharTok{|}\NormalTok{Estacion,}\AttributeTok{method=}\StringTok{"REML"}\NormalTok{,}\AttributeTok{data=}\NormalTok{leon\_tb\_l)}
\FunctionTok{summary}\NormalTok{(m1d\_final)}\CommentTok{\#Diferencias entre 2017 y 2016 respecto de 2014}
\end{Highlighting}
\end{Shaded}

\begin{verbatim}
## Linear mixed-effects model fit by REML
##   Data: leon_tb_l 
##        AIC       BIC   logLik
##   -1286.37 -1251.946 653.1851
## 
## Random effects:
##  Formula: ~1 | Estacion
##         (Intercept)   Residual
## StdDev:  0.01070344 0.01074035
## 
## Fixed effects:  Peso_medio ~ 1 + año 
##                  Value   Std.Error  DF  t-value p-value
## (Intercept)  1.0818501 0.002338837 143 462.5591  0.0000
## año2015     -0.0106075 0.003451569 143  -3.0732  0.0025
## año2016     -0.0030930 0.006149433 143  -0.5030  0.6158
## año2017     -0.0075608 0.002460199 143  -3.0733  0.0025
## año2018     -0.0060582 0.003208584 143  -1.8881  0.0610
## año2019     -0.0077743 0.006149433 143  -1.2642  0.2082
## año2020     -0.0074722 0.002428914 143  -3.0763  0.0025
## año2021     -0.0091514 0.003178765 143  -2.8789  0.0046
##  Correlation: 
##         (Intr) añ2015 añ2016 añ2017 añ2018 añ2019 añ2020
## año2015 -0.621                                          
## año2016 -0.293  0.264                                   
## año2017 -0.568  0.397  0.233                            
## año2018 -0.672  0.638  0.283  0.427                     
## año2019 -0.293  0.264  0.237  0.233  0.283              
## año2020 -0.630  0.432  0.245  0.540  0.465  0.245       
## año2021 -0.680  0.644  0.284  0.431  0.695  0.284  0.469
## 
## Standardized Within-Group Residuals:
##         Min          Q1         Med          Q3         Max 
## -2.35920539 -0.48285040 -0.01712622  0.49199585  2.70961349 
## 
## Number of Observations: 239
## Number of Groups: 89
\end{verbatim}

El modelo final lo utilizamos patra contrastar los diferentes niveles de
la variable predictora, en este caso los diferentes años mediante
comparaciones múltiples mediante eltest de Tukey. Comrobamos el modelo
de forma gráfica

\includegraphics{Informe_files/figure-latex/unnamed-chunk-47-1.pdf}
\includegraphics{Informe_files/figure-latex/unnamed-chunk-47-2.pdf}
\includegraphics{Informe_files/figure-latex/unnamed-chunk-47-3.pdf}
\includegraphics{Informe_files/figure-latex/unnamed-chunk-47-4.pdf}

\begin{verbatim}
## 
##   Simultaneous Tests for General Linear Hypotheses
## 
## Multiple Comparisons of Means: Tukey Contrasts
## 
## 
## Fit: lme.formula(fixed = Peso_medio ~ 1 + año, data = leon_tb_l, 
##     random = ~1 | Estacion, method = "REML")
## 
## Linear Hypotheses:
##                    Estimate Std. Error z value Pr(>|z|)  
## 2015 - 2014 == 0 -1.061e-02  3.452e-03  -3.073   0.0370 *
## 2016 - 2014 == 0 -3.093e-03  6.149e-03  -0.503   0.9996  
## 2017 - 2014 == 0 -7.561e-03  2.460e-03  -3.073   0.0371 *
## 2018 - 2014 == 0 -6.058e-03  3.209e-03  -1.888   0.5223  
## 2019 - 2014 == 0 -7.774e-03  6.149e-03  -1.264   0.8976  
## 2020 - 2014 == 0 -7.472e-03  2.429e-03  -3.076   0.0369 *
## 2021 - 2014 == 0 -9.151e-03  3.179e-03  -2.879   0.0653 .
## 2016 - 2015 == 0  7.514e-03  6.205e-03   1.211   0.9169  
## 2017 - 2015 == 0  3.047e-03  3.349e-03   0.910   0.9822  
## 2018 - 2015 == 0  4.549e-03  2.841e-03   1.601   0.7200  
## 2019 - 2015 == 0  2.833e-03  6.205e-03   0.457   0.9998  
## 2020 - 2015 == 0  3.135e-03  3.252e-03   0.964   0.9752  
## 2021 - 2015 == 0  1.456e-03  2.808e-03   0.519   0.9995  
## 2017 - 2016 == 0 -4.468e-03  6.069e-03  -0.736   0.9950  
## 2018 - 2016 == 0 -2.965e-03  6.077e-03  -0.488   0.9996  
## 2019 - 2016 == 0 -4.681e-03  7.595e-03  -0.616   0.9984  
## 2020 - 2016 == 0 -4.379e-03  6.034e-03  -0.726   0.9954  
## 2021 - 2016 == 0 -6.058e-03  6.067e-03  -0.999   0.9698  
## 2018 - 2017 == 0  1.503e-03  3.098e-03   0.485   0.9997  
## 2019 - 2017 == 0 -2.135e-04  6.069e-03  -0.035   1.0000  
## 2020 - 2017 == 0  8.865e-05  2.344e-03   0.038   1.0000  
## 2021 - 2017 == 0 -1.591e-03  3.068e-03  -0.518   0.9995  
## 2019 - 2018 == 0 -1.716e-03  6.077e-03  -0.282   1.0000  
## 2020 - 2018 == 0 -1.414e-03  2.991e-03  -0.473   0.9997  
## 2021 - 2018 == 0 -3.093e-03  2.495e-03  -1.240   0.9068  
## 2020 - 2019 == 0  3.021e-04  6.034e-03   0.050   1.0000  
## 2021 - 2019 == 0 -1.377e-03  6.067e-03  -0.227   1.0000  
## 2021 - 2020 == 0 -1.679e-03  2.959e-03  -0.567   0.9990  
## ---
## Signif. codes:  0 '***' 0.001 '**' 0.01 '*' 0.05 '.' 0.1 ' ' 1
## (Adjusted p values reported -- single-step method)
\end{verbatim}

Podemos ver en la tabla de contrastes que para los datos de peso medio
existen diferencias significativas de los datos del año 2014, solo con
respecto a 2015. No obstante los datos parecen indicar una reducción
ligera del peso medio probablemente debido a que la pesca sin muerte en
los tramos libres ha supuesto un aumento apreciable de biomasas pero aún
más las densidades con lo que el peso medio puede mostrar una ligera
reducción.

\hypertarget{discusiuxf3n}{%
\section{3. DISCUSIÓN}\label{discusiuxf3n}}

De los resultados obtenidos para la provincia de León que es la
provincia con mayor número de estaciones de muestreo, se observa que
tanto para las variables biomasa como para la densidad esta aumenta en
todos los años en comparación con el año base (2014) aunque los efectos
son significativos para los años 2017, 2019, 2020 y 2021 en el caso de
la biomasa de trucha y para los años 2017, 2018, 2019, 2020 y 2021 en el
caso de la densidad.

Debidoa la falta de datos ambientales, tales como factores hidrológicos,
composición y estructura del hábitat que caractericen las esptaciones es
imposible profundizar en parámetros excplicativos. Igualmente al no ser
publicados las estructuras de las poblaciones de trucha inventariadas
tampoco conocemos los parámetros poblacionales principales que nos
ayuden a comprender la situación en profundidad. Mortalidad, crecimiento
o producción, parámetros éstos relevantes para una mejor comprensión de
las dinámicas poblacionales no es posible determinarlas al no haber sido
publicadas las existencias por cohortes de las poblaciones
inventariadas. No obstante el análisis de densidades, biomasas y en
menor mediada el peso medio, entre los diferentes años desde el 2014,
parecen claros al mostrar los datos un importante incremento en los
valores de la biomasa de trucha y sobre todo en las densidades, que han
visto multiplicado por 3 los valores de 2014 y por 2 en el caso de la
biomasa. La variable peso medio, sufre una ligera disminución ya que
siendo determinada como el cociente entre la biomasa unitaria y la
densidad unitaria, y debido a que el aumento de la densidad ha sido
superior al aumento en biomasa, este valor ha disminuido aunque no de
forma significativa.

Para analizar con más profundidad estos datos sería relevante determinar
para las estaciones de muestreo la estructura de cohortes, parámetros de
crecimiento y mortalidad. Con ello se podría determinar con más
precisión el alcance de estos valores de peso medio.

Sería importante recabar datos anteriores a 2014, ya que pese a que se
ha tomado como año de comparación para comprobar los efectos de la
declaración de los libres como libres sin muerte, debido a que os
muestreos se realizaron al final de la temporada de pesca, es posible
que en el año 2014 también se hayan notado los efectos de la entrada en
vigor de la ley, y se hayan reclutado en las tallas pescables individuos
que no han sido objeto de extracción, y por ello se hayan atenuado las
diferencias entre los grupos.

\end{document}
